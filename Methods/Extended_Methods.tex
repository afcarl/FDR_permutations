% Created 2016-07-29 Fri 14:38
\documentclass{article}
\usepackage[utf8]{inputenc}
\usepackage[T1]{fontenc}
\usepackage{graphicx}
\usepackage{grffile}
\usepackage{longtable}
\usepackage{wrapfig}
\usepackage{rotating}
\usepackage[normalem]{ulem}
\usepackage{amsmath}
\usepackage{textcomp}
\usepackage{amssymb}
\usepackage{capt-of}
\usepackage{hyperref}
\usepackage{algorithm}
\usepackage{algorithmic}
\author{Daniel Kessler, Michael Angstadt, Chandra Sripada}
\date{\today}
\title{Extended Methods}
\hypersetup{
 pdfauthor={Daniel Kessler, Michael Angstadt, Chandra Sripada},
 pdftitle={Extended Methods},
 pdfkeywords={},
 pdfsubject={},
 pdfcreator={Emacs 24.3.1 (Org mode 8.3.5)}, 
 pdflang={English}}
 
\newcommand{\subtext}[2]{
#1_{_{\text{#2}}}
}
 
\begin{document}

\maketitle
%\tableofcontents

\section{Data \& Code}
This document, along with source code and data for all analyses, is available at \url{http://github.com/mangstad/FDR_permutations}. 
Anders Eklund graciously shared with us the per-participant "first-level" maps as well as group-level results for each contrast analyzed as part of their study  \cite{eklund_cluster_2016}. These contrasts were in turn derived from two studies \cite{duncan_consistency_2009,tom_neural_2007}, which were available at openfMRI \cite{poldrack_toward_2013}. 

%All data was originally obtained from openfMRI \cite{poldrack_toward_2013}, which served as repository for two original studies \cite{duncan_consistency_2009,tom_neural_2007}. 
%Anders Eklund graciously shared with us the per-participant "first-level" maps for each contrast from each study as well as group-level results \cite{eklund_cluster_2016}. This document, along with source code and data for all analyses, is available at \url{http://github.com/mangstad/FDR_permutations}.

\section{Overview}
We consider each contrast from each study separately and perform the following steps. 
In order to generate a realization of the null, we randomly flip the sign of each participant's contrast map, conduct a voxelwise, one-sample t-test across participants, apply a cluster defining threshold (CDT) of \{.001,.01\}, and then record the voxel extents of all clusters that emerge (identified as connected components using the FSL-standard 26-degree connectivity criteria that considers voxels connected if they share any face, edge, or vertex). 
We repeat this procedure 5,000 times to obtain an empirical mass function of cluster sizes under the null. 
To obtain a per-cluster uncorrected \textit{p}-value, we compare the extent for each cluster that arises from the unflipped design to the contrast's null distribution. 
Finally, we submit the vector of uncorrected \textit{p}-values to an FDR procedure \cite{benjamini_controlling_1995} with $\alpha = .05$.

\section{Cluster Extent Test Statistic and Procedure}
We desire to assess the statistical significance of each of our clusters.
To accomplish this, we select any one of these clusters (at random), and consider the distribution of its corresponding test statistic, $Z$:
\begin{equation}
  Z = 
  \begin{cases}
    \text{Extent of randomly selected cluster}, & nCluster\geq1\\
    0, & \text{else}
  \end{cases}
\end{equation}


Let $N$ be the number of participants under consideration and $I, J, K$ represent the number of voxels in the three dimensions. In order to obtain $Z$, we require the following: $X$, a collection of $N$ three-dimensional statistical parameter maps, one per participant; $Y$, a vector of signs/labelings/covariates corresponding to each participant (because all studies considered here use a one-sample testing framework, all elements of $Y$ are $1$); and $CDT$, a cluster defining threshold (a \textit{p}-value).

We can repeat the below procedure multiple times for the observed labelings until $Z$ corresponds to each of our clusters.

\begin{algorithm}[H]
  \caption{Return Extent of Arbitrary Cluster}
  \begin{algorithmic}[1]
    \REQUIRE $X \in \mathbb{R}^{I \times J \times K \times N}$
    \REQUIRE $Y \in \mathbb{R}^{N}$ \text{; In current setting, } $Y \in \{-1,+1\}^{N}$
    \REQUIRE $0 < CDT < 1$
    \FOR{\text{Each voxel}}
    \STATE Fit a GLM, $X_{i,j,k} \sim Y$, return $p$-value
    \IF{$p < CDT$}
    \RETURN{$p$}
    \ELSE
    \RETURN{$0$}
    \ENDIF
    \ENDFOR
    \STATE Clusterize, with 26-degree connectivity, thresholded map
    \IF{$nClusters \geq 1$}
    \RETURN{Extent of randomly selected cluster}
    \ELSE
    \RETURN{$0$}
    \ENDIF
  \end{algorithmic}
\end{algorithm}

\section{Permutation Strategy}
Under the null hypothesis, the values of $Y$ are independent of $Z$ for our data $X=x$. We can thus realize the null distribution by repeatedly altering $Y$ and observing a new $Z$. Ultimately, we arrive at $p$-values by comparing the $z$ obtained from the true labelings $y$ with the distribution of $Z$ obtained from altered versions of $Y$. In the one-sample t-test case (the one studied here), all $Y$ are $1$, so we alter $Y$ by flipping the sign of each element at random. An exact test would visit all $2^N$ possible flippings of the signs of the elements of $Y$, but for computational reasons, we approximate this by randomly visiting a single possible flipping, and repeating this process 5,000 times.

\section{Permutation Aggregation \& PMF}
We are ultimately interested in $P(Z>z|X=x)$. Importantly, $Var \big[ Z|X=x,Y=y \big] > 0$  when $nCluster>1$ and there are 2 or more unique cluster extents because $Z$ is a randomly selected from among $nCluster$ options.
While we could use repeated realizations of $Z$ for a given $x,y$ to approximate $P(Z|X=x,Y=y)$ we can, instead, directly obtain this distribution as the probability mass function (i.e., the normalized histogram of cluster extents) for a given $Y$. This enables us to visit each $Y$ only once. Because visiting any $Y$ is equiprobable under our sampling scheme, $P(Z|X=x)$ is simply the average of $P(Z|X=x,Y=y)$ for all $y$ visited.

\section{Clusterwise, Uncorrected \textit{p}-Values}

We begin with the extents of all clusters observed under the true labels (not only those previously reported which were restricted to those significant at $\subtext{\alpha}{RFT-FWE}<.05$). 
For each cluster, we assign the observed extent to $z$ and calculate $P(Z>z|X=x)$ (using the null distribution arrived at above) and retain this value as the uncorrected \textit{p}-value for that cluster.

\section{Multiple Testing Correction: FDR}

While the uncorrected \textit{p}-values from above were obtained using a nonparametric rather than parametric procedure, we follow the approach from Chumbley and Friston \cite{chumbley_false_2009}: We submit the uncorrected \textit{p}-values as a vector to the FDR controlling procedure established in Bejamini and Hochberg \cite{benjamini_controlling_1995}, with $\alpha_{FDR}=.05$, retain the $\subtext{q}{FDR}$-value, and record whether the null hypothesis ought to be rejected (significant value) or not. As indicated above, we are certainly not the first to apply FDR to cluster properties; cf. \cite{chumbley_false_2009,heller_cluster-based_2006,chumbley_topological_2010}.

\section{Comparing $\subtext{p}{RFT-FWE}$ with $\subtext{q}{FDR}$}
In Figure 1, we plot $\subtext{p}{RFT-FWE}$ vs $\subtext{q}{FDR}$ on a $-log_{10}$ scale for all clusters significant at $\subtext{\alpha}{RFT-FWE}=.05$.
We emphasize (with shading) those clusters with corresponding $\subtext{q}{FDR}\leq .05$, i.e., the clusters that remain significant under $\subtext{\alpha}{FDR}=.05$. 


\bibliography{../Letter/biblio}
\bibliographystyle{../Letter/pnas2011}
\end{document}